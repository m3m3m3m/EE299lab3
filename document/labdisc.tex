\subsection{Brief Design Specification}
\subsubsection{Overall summary description of the module}
This module is a box-guessing game designed by ourselves. The input is received from a joystick, four direction buttons and one tilt button. The user can control and choose boxes with joystick and direction buttons. The user can also turn the screen off by flipping the tilt button. The game screen is a LCD screen connected to one of the two Arduino boards. The user's command and movement will be transfered from one board to another and will be reflected on the serial monitors.

\subsubsection{Specification of the public interface to the module}
\paragraph{Inputs, Outputs and Side Effects}
\hfill \newline
Since there are two boards in our module, we will introduce the two boards separately.
\subparagraph{Master}
\hfill \newline
The input of the master board is a joystick(containing four directions), and four direction buttons. These inputs are received by passing pointers to functions that realize the reactions to different buttons and directions of joystick movement to the public interface of the joysticks and buttons.Users can use joystick to move the cursor to select the box they want to guess before the game starts and the correct box they think in the open box phase. They can also use the joystick to modify the settings. To enter the setting page, they will need to push the up button before the game starts. They can push the right button to confirm after they choose the box to guess, before and after they select the box as their guess, after they finish setting, and after they finish reading the tutorial. They can push the down button before the game starts to read the tutorial and to exit the tutorial page. \\
\hfill \newline
Their commands will be transfered to the slave board through serial port, so the output of the master board would be serial that can be checked on the serial monitor on PC. It will send a code number for each commands followed by nothing or some digits or string to be delivered. \\
\hfill \newline
Each time the user moves the joystick or pushes a button, some parameters will be changed to help to compute the new display of the screen.
\subparagraph{Slave}
\hfill \newline
The input of the slave board is from the serial port. At the beginning of each time of transmission, a digit code for the type of command will be sent. According to the type of the command, different number of digits or string will be sent. A tilt button also serves as one part of the slave board's input.\\
\hfill \newline
The output of the slave board is its LCD screen and a buzzer. Also, it will print what it will do on the serial monitor after each movement.\\
\hfill \newline
Depending on the command it receives from the serial port and the tilt button, it will change different local variables such as the number of boxes, the speed of swapping, the cursor position, etc.

\paragraph{Pseudo English description of algorithms, functions, or procedures}

\subparagraph{Buzzer}
\hfill \newline
We used a buzzer to play a sound effect after user's every single guess. Two melodies are written in the code in pitches and durations, for a correct answer and a wrong one. To play a music, function tone() is called to make a sound of a designated pitch, and then it delays until the tone ends. The delay time are set 30 percent longer than the durations of tones to make them sound like seperate notes. This repeats until all the notes are played.

